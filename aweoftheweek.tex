\documentclass[12pt,oneside,landscape]{memoir}

%%%%%%%%%%%%%%%%%%%%%%%%%%%%%%%%%%%%%%%%%%%%%%%%%%%%%%%%%%%%%%%%%%%%%%%%%%%%%%%%%%%%%%%%
%
% Altered World Event of the Week is a Monster of the Week derivative.
%
% Much of the textual content in the playbooks is copied from the original playbooks,
% following approval by Michael Sands for the free distribution of the AWEotW document.
% See: https://genericgames.co.nz/third_party_policy/ for his third party policy.
%
% Re-use of this script is for personal reproduction of the formatting and generation
% of other custom playbooks.  Any reproduction of the text should seek approval per the
% aforementioned generic games third party policy.
%
% Monster of the Week is copyrighted by Evil Hat Productions, LLC and Generic Games.
%
%%%%%%%%%%%%%%%%%%%%%%%%%%%%%%%%%%%%%%%%%%%%%%%%%%%%%%%%%%%%%%%%%%%%%%%%%%%%%%%%%%%%%%%%

% References used for creation of this document:
% https://github.com/exposit/dw-min-template-latex
% https://www.overleaf.com/latex/templates/dungeon-world-playbook-template/trprzrmbfzry

% package imports
\usepackage[utf8]{inputenc}
\usepackage[T1]{fontenc}
\usepackage{xcolor}
\usepackage{multicol}
\usepackage{tcolorbox}
\usepackage{enumitem}
\usepackage{titlesec}
\usepackage{tikz}
\usepackage{aweotwplaybook}

% formatting
\usepackage[bindingoffset=0mm,heightrounded]{geometry}

% import me last
\usepackage[colorlinks=true]{hyperref}

% ---------------------------------------------------------------
\begin{document}

% overall formatting
\pagestyle{empty}
\pagecolor{black}
\color{white}
\setlength{\parindent}{0pt}
\setlength{\parskip}{0pt}
\setlength\columnsep{8mm}
%
\hypersetup{linkcolor=blue,urlcolor=cyan}
\urlstyle{same}
%
\titleformat%
{\chapter}%
[display]%
{\bfseries\LARGE}%
{\thechapter}%
{0mm}%
{\vspace{-8mm}}%
[\vspace{-8mm}]%
%
\tcbset{
    colframe=blue!50!white,
    colback=black,
    coltext=white,
    top=4mm,
    bottom=8mm,
    right=8mm,
    left=4mm,
}
%
\setlist{
    leftmargin=8mm,
    itemsep=0mm,
    parsep=1mm,
    partopsep=0mm,
}
%
% custom commands
\newcommand{\goof}[2]{\textbf{\normalsize #1}\\[4mm]{\footnotesize #2}}
\newcommand{\goofcat}[1]{{\small #1}}
\newcommand{\whitespace}[1]{\phantom{.}\\[#1mm]}
\newcommand{\pboverview}[2]{\textbf{\normalsize #1: }{\footnotesize #2}\\[2mm]}

% Title ---------------------------------------
\makeatletter
\renewcommand{\maketitle}{
\begin{center}

\phantom{.} % necessary to add space on top before the title
\vspace{1cm}

\textbf{\Huge \@title}
\vspace{1cm}

{\Large \@author}\\[0.25cm]

{\large \@date}

\vspace{1.5cm}
\begin{tikzpicture}
% inner
\draw[white, ultra thick] (-1.5,2.5) -- (2.5,-5);
\draw[white, ultra thick] (2.5,-5) -- (6.5,2.5);
% outer
\draw[white, ultra thick] (-2.5,2.5) -- (2.5,-5);
\draw[white, ultra thick] (2.5,-5) -- (7.5,2.5);
% top
\draw[white, ultra thick] (-2.5,2.5) -- (7.5,2.5);
\end{tikzpicture}

\vspace{1cm}
If you have questions or comments,\\feel free to contact me at \href{mailto:innumerable.engines@gmail.com}{innumerable.engines@gmail.com}\\[1cm]
Derived from Monster of the Week by Michael Sands.\\
Approved for use with Monster of the Week.\\
Monster of the Week is copyrighted by Evil Hat Productions, LLC and Generic Games.\\
\end{center}
}\makeatother

\title{Altered World Event of the Week}
\author{Innumerable Engines Games}
\date{Version Beta-2020.11.28}
\maketitle
\pagebreak

% Intro ---------------------------------------
\chapter*{What Is This?}
\begin{multicols}{2}
AWE of the Week is a Monster of the Week derivative providing a natural theme and atmosphere for games set in the SCP Foundation or the world of Remedy’s 2019 game, Control.  It focuses on defining the theme, running the game with Altered World Events at the heart of each mystery, and providing a host of playbook changes that are tailored to match the atmosphere.
\\[4mm]
This is not, by any measure, a rewrite of the Monster of the Week manual.  The biggest textual changes you will find are the retooled playbooks.  There’s a couple updates to basic moves, but the rest is just suggestions for flavor.  If I didn’t mention something from the manual here, assume it didn’t need to change.
\\[4mm]
If you’re already playing Monster of the Week, most of the content should feel familiar.  The content provided is not altogether new, but rather a spotlight on the proposed theme generated with a mix of new and old.  You could play vanilla MotW and still produce the intended experience (especially if you’re using the Tome of Mysteries supplement), but MotW paints its world in broad strokes, far outside the frame of paranormal government agencies, and you’ll have to contend with those differences as you go.  If you’re already planning for this specific theme, might as well start here.
\\[4mm]
If you haven’t played Monster of the Week yet, but you’re looking to play a SCP or paranormal-agency game, then you’re in the right place.  MotW is already well suited, both descriptively and mechanically, to stories about government agents investigating weirdness in the world. This supplement will help you bring that theme to life.  However, this is not a full game in itself.  You’ll still need to purchase, read, and understand the vanilla game to get the full experience.  There’s a link for that on the last page.

\begin{tcolorbox}
\section*{In Brief}
\begin{itemize}[leftmargin=*,parsep=4mm]

\item A general theme conversion: the Paranormal replaces Magic, players are Bureau Agents instead of Hunters, new Playbooks fit character archetypes and settings from Control and SCP.

\item The episodic Monster is replaced with supernatural AWEs: altered items, paranormal influence, dimensional clashes (and possibly a monster now and then, too).  

\item A reduction in combat focus.  Mysteries are less focused on destroying a threat and more about understanding, solving, or containing the situation.

\item Intended for use with, not a replacement of, the MotW core text.

\end{itemize}
\end{tcolorbox}

\end{multicols}
\pagebreak
%
%
% Theme and Setting ---------------------------------------
%
\chapter*{Theme, Changes, and Altered World Events}
\begin{multicols}{2}

\section*{The Setting}
The setting of this supplement originates within government bureaucracy.  The players’ team is not a rag-tag, tribal, or artisanal outfit of the few people who recognize the unnatural and the dangerous running amok in the world.  This is their day job.  A nine-to-five grind within an organization that is aware of, or at least claims to be aware of, all the supernatural goings-on in the world and that strives to keep it in check.  The Bureau’s job is to help everyone else remain blissfully blind.  After all, the truth would hurt their brains very badly.
\\[4mm]
Being a government employee means you’re no longer a Hunter; no longer an itinerant and unbound pursuer of the biggest and most dangerous game that exists.  You’re an Agent of the Bureau.  You have bosses and salary, paperwork and protocol.  You clock in and clock out of the office when you come and go.  They require you to save receipts when you work abroad so they can document the expenditures in a spreadsheet.  The routine isn’t all that bad.  It’s a savory bit of plainness wrapped around the weirdness waiting in the laboratories, or in the cross country trips to find out why one specific subway car lets people inside of it but never out again, or how someone came to own a polaroid camera where you can reach into the pictures and pull out whatever is shown within.
\\[4mm]
The office that you work from is a paranormal entity all on its own.  A living, shifting labyrinth of seemingly endless size, only sometimes content to get wrestled into a set of cubicles; a collision of dimensional intersections; a prison of everyday objects grown hungry with appetites and desires.  If you walk out the bathroom on a different floor than you entered or find your coffee cup brackish with water that will never fully pour out… well, it wouldn’t be the first time.  Like we said: a little bit of plain, a whole lot of weird.

\section*{Differences and Similarities}
Playing this derivative means you’re looking to make some changes about the tenets of the universe as established in Monster of the Week.  The expected setting was already introduced above.  The rest of the differences are about what MotW is, and what this supplement isn’t:  MotW is about, well, monsters.  This setting is about the paranormal.  MotW likes magic.  This setting likes fringe science.  MotW drives the play towards combat.  This setting prefers exploration and hazardous containment.  MotW invites every character archetype to the party.  This setting has a tight-knit friend group who all share the same interests and want to keep it that way.  At large, MotW is a generalized breadth of content you can tailor to match all manner of tropes and pop culture.  This setting has very specific goals in mind for the style of game at play.
\\[4mm]
The rest of it remains the same.  The power level of the playbooks, the level of danger in the world, and the Keeper’s motives should match across both experiences.  Harm and healing occur with the same severity and consequences (arguably, due to the reduced focus on combat, when the Keeper does hit the Agents, they can hit even harder).  Most importantly the pacing of MotW remains: episodic mysteries with the occasional extended arc.
\\[4mm]
Finally, the new playbooks are a mix of existing and custom text.  MotW provided a wealth of moves with the appropriate tone and action and many of those are copied over from their original form (or something similar to it, given some thematic changes) to the new books.  However, these moves don’t always land in the same place.  The new books often mix-and-match existing moves to produce a unique archetype, one that is more fitted for the setting at hand.

\section*{Episodic AWEs}
Altered World Events take the place of Monsters in this setting.  That doesn’t mean monsters won’t appear in your game, just that they're not the weekly focus.  Most times the mystery will revolve around something weird, unpredictable, alien, or altered.  Whether the situation is simply unnatural or actively hostile the Agents’ primary goal is not to destroy the paranormal source of the event, but to secure and contain it.
\\[4mm]
What makes some event a notable AWE as opposed to a regular occurence?  The presence of the paranormal counts for a lot.  However, that's only one part of the puzzle.  The real thing requires adherence to two principles:  First, something in the world has appeared which itself behaves, or causes the rest of the world to behave, in a way that it shouldn’t.  Second, the Bureau wants to intervene as soon as possible, minimize collateral, keep the media out of it, and cover up anything that leaks.  
\\[4mm]
Michael Sands has already written a fantastic guide for what makes a good, paranormal-oriented mystery in \href{https://genericgames.co.nz/files/MotW_more_weirdness.pdf}{MotW More Weirdness}.  Look for the “Phenomena” section.  If you’re trying to run this supplement and haven’t read that, or looked into the Tome of Mysteries expansion, I highly recommend you follow the advice there first.  It provides all the tools for designing a good, monster-less mystery, and you can run all sorts of AWEs just following that design.  What I’ve listed aside here are a couple further suggestions for specific themes and flavors.
\whitespace{8}

\begin{tcolorbox}[bottom=4mm]
\section*{Types of AWEs}
\begin{itemize}[leftmargin=*,parsep=1mm]

\item \textbf{Altered Item} (motivation: to break natural laws, receive worship, and escape containment).
\item \textbf{Teleporter} (motivation: to transport people or things to dangerous places).
\item \textbf{Alter-dimensional Being} (motivation: to be unfathomable).
\item \textbf{Astral Being} (motivation: to possess and destabilize Altered Items)
\item \textbf{Object of Power} (motivation: to grant someone more power than they can handle).
\item \textbf{Disease} (motivation: to spread and thrive or consume).
\item \textbf{Mold} (motivation: to transform creatures, people, and things).
\item \textbf{Corruption} (motivation: to warp or take control of creatures and people).
\item \textbf{Resonance} (motivation: to overwhelm, control, or drive to insanity).
\item \textbf{Fringe Experiment} (motivation: to unleash dangers).
\item \textbf{Bureau Experiment} (motivation: to trifle with powers beyond control).
\item \textbf{Thresholds} (motivation: to introduce entities and phenomena into the world).
\item \textbf{Portals} (motivation: to take people to dangerous or exotic places).
\item \textbf{House Shifts} (motivation: to reconfigure the house unexpectedly).
\item \textbf{Pocket Dimension} (motivation: to trap or to hide things away).
\item \textbf{Subspace} (motivation: to impose different laws and logic within).

\end{itemize}
\end{tcolorbox}

\end{multicols}
\pagebreak
%
%
% Basic Moves ---------------------------------------
%
\chapter*{Basic Move Updates}
\begin{multicols}{4}
\begin{pbsect}{Investigate a Mystery}[]
When you \textbf{investigate a mystery}, roll +Sharp. On a 10+ hold 2, and on a 7-9 hold 1. One hold can be spent to ask the Keeper one of the following questions:
\holdoptions%
    {What happened here?,
    What could do something like this?,
    Are we in danger right now?,
    How difficult will it be to contain?,
    Where did it go?,
    Who is it connected to?,
    What is being concealed here?}
\end{pbsect}
%
\begin{pbsect}{Fringe Science (replaces Big Magic)}
When you want to \textbf{create or adapt a device} to analyse, deal with, or produce a paranormal phenomenon, tell the Keeper what you want to do and roll +Weird.  On a 10+ pick two guarantees.  On a 7-9, pick one.
\br[2mm]
%
Advanced: 12+ take all three guarantees, or two guarantees and the Keeper adds another benefit.
\end{pbsect}%
\br[2mm]
%
\begin{blurb*}{The Keeper may choose one or more requirements:}%
\holdoptions%
    {{You’ll have to spend a lot of time (days or weeks) doing research.},
    {You need to experiment first; expect lots of failures before you get it right.},
    {You need some rare and weird ingredients, supplies, or equipment.},
    {After it starts up, it’ll take time before it reaches the full effect.},
    {It requires huge amounts of power or fuel.},
    {You need a lot of people to help out.},
    {The ritual needs to be done at a particular place and/or time.},
    {It will have a specific side-effect or danger.}}
\end{blurb*}
%
\begin{blurb*}{Then you pick your guarantees:}
\holdoptions%
    {{Once started, it cannot be stopped easily.},
    {It does exactly what you intended, and only what you intended.},
    {It doesn’t draw attention or make you any more obvious than you already are.}}
\end{blurb*}
%
\begin{pbsect}{Ritual (replaces Use Magic)}
When you \textbf{perform a ritual to invoke or placate the paranormal}, say what you’re trying to achieve and how you do it, then roll +Weird.  10+ it worked without issues, choose your effect.  7-9: It worked imperfectly.  Choose one effect and one glitch.
\brln
%
Advanced: 12+ the Keeper adds some benefit.
\end{pbsect}
%
\begin{blurb*}{Effects:}
\holdoptions%
    {{Bar a place or portal to a specific person, item, or entity.},
    {Trap a person, altered item, or entity.},
    {Calm or satisfy a paranormal item or entity.},
    {Communicate with something that you do not share a language with.},
    {Observe another place or time.}}
\end{blurb*}
%
\begin{blurb*}{Glitches:}
\holdoptions%
    {{The effect is weakened.},
    {The effect has a short duration.},
    {You take 1-harm psychic ignore-armor.},
    {The ritual draws immediate, unwelcome attention.},
    {There’s a problematic side effect.}}
\end{blurb*}

\begin{blurb*}{The Keeper may say that…}
\holdoptions%
    {{The ritual requires rare or weird materials.},
    {The ritual will take extra time to perform.},
    {The ritual requires you to focus your attention on \rule{0.3\linewidth}{0.4pt}.},
    {The ritual requires you to set up some equipment.},
    {You need one or two more people to help you out.},
    {You need to refer to some prior research for the details.}}
\end{blurb*}

\end{multicols}
\pagebreak

% Playbook Overview ---------------------------------------
\chapter*{Bureau Agents Overview}

\begin{multicols}{3}

\begin{tcolorbox}[bottom=4mm,after skip=8mm]
\section*{An Agent's Agenda}
\begin{itemize}

\item Act like you’re the only one in the Bureau who can solve these problems (because you are).
\item Find the altered world events and contain them.
\item Play your agent like they’re a real person.

\end{itemize}
\end{tcolorbox}

\pboverview{The Director}
{A close port of the Chosen.  The signature weapon made a great parallel to the service weapon.  Further, the chosen was already themed around their importance.  The design leans more toward playing Trench than Jesse, since Jesse is so many other things that became their own books as well (the outsider, the psychic, etc).  This means you’re more strictly “the boss” than “the savior”.}

\pboverview{The Investigator}
{Started off as the Flake, with the end result mixing in flavor from the Searcher, too.  The investigator isn’t as focused on being the conspiracy nut in this setting.  It’s more about life as a rank and file paranormal detective.  In Control these are the agents you read about in the various reports scattered throughout the oldest house.  The ones who get flown out here and there to do the grunt work of figuring out whether a suspected AWE is truly paranormal, or just another loony daydream.}

\pboverview{The Ranger}
{The Ranger was the easiest book to transition over, it’s almost a 1:1 replica of the Professional; that book has a perfect fit for the highly trained Ranger operatives in control.  The professional’s Agency mechanics were dropped, since all players are involved with the Bureau already.  In replacement, you get a squad of other rangers at your back.  As you might expect from its source, this book is far and away the most combat oriented and militaristic, especially relative to the toned back emphasis on combat.}

\pboverview{The Janitor}
{The Janitor is one of the two Weird-focused playbooks in the group, and is one of the most custom designed of the set (IE, not simply a reshuffling of other moves).  It has more focus around being inside the oldest house than the other books.  As a result, if your game focuses on space outside the building, it might make less sense to include this one.  In trade, janitors have a much greater pull on the narrative about what the oldest house contains, and what can be done within it.}

\pboverview{The Researcher}
{What kind of bureau would it be without a host of lab techs?  The Researcher gets most of its pattern from the Expert, since the Haven made such an easy transition into a specialized lab to work inside.  Like the Janitor, they’re a little more focused on staying inside the oldest house.  However, they straddle the divide more easily, and it shouldn’t be difficult to include them in a game where events take place both inside and outside the house.}

\pboverview{The Outsider}
{Not everyone involved in the Bureau is standard government goon.  The Outsider is the Wronged’s bad luck of getting forced into this world outside their wants (minus all the emotional baggage), plus the Flake’s conspiratorial nature (minus all the proper detective work).  They’re the book for someone who doesn’t want to drink the Bureau’s kool-aid, or whose character might not even be aware of the Bureau when all of this starts.}

\pboverview{The Psychic}
{The Psychic is the effective “magic user” of the bunch, and draws largely from the Spooky and the Divine.  While everyone else simply works a day job in a paranormal environment, the psychic directly engages with, and is driven by, their connection to a paranormal entity.  If you want to play as Jesse Faden, and you aren’t happy with the Trench-oriented Director playbook, this is the one for you.  Give yourself a resonance, fling some TVs around with your mind, and don’t mind the growing chant of voices; it’s all perfectly under your control.}

\end{multicols}
\pagebreak

% Setting font size for playbooks...
\fontsize{10}{12}\selectfont

% The Director ---------------------------------------
%%%%%%%%%%%%%%%%%%%%%%%%%%%%%%%%%%%%%%%%%%%%%%%%%%%%%%%%%%%%%%%%%%%%%%%%%%%%%%%%%%%%%%%%
%
% Altered World Event of the Week is a Monster of the Week derivative.
%
% Much of the textual content in the playbooks is copied from the original playbooks,
% following approval by Michael Sands for the free distribution of the AWEotW document.
% See: https://genericgames.co.nz/third_party_policy/ for his third party policy.
%
% Re-use of this script is for personal reproduction of the formatting and generation
% of other custom playbooks.  Any reproduction of the text should seek approval per the
% aforementioned generic games third party policy.
%
% Monster of the Week is copyrighted by Evil Hat Productions, LLC and Generic Games.
%
%%%%%%%%%%%%%%%%%%%%%%%%%%%%%%%%%%%%%%%%%%%%%%%%%%%%%%%%%%%%%%%%%%%%%%%%%%%%%%%%%%%%%%%%

% the director ---------------------------
%
% -- front page
%
\playbookpage{%
%
% -- pg.1/2 col.1/3
%
\pbcommon{Director}%
{I didn't want the job of Director, didn't want the power.  But there were no other candidates.  Only me, always in the background, working my ass off.}%
{When you spend a point of luck, the Keeper will bring your duties into play.}%
%
}{% -- pg.1/2 col.2/3
%
\begin{pbsect}{THE DIRECTOR'S DUTIES}[]
You get to decide what your position has in store for you.  Pick how you were selected as Director of the Bureau, and what The Board and The Bureau are expecting of you, on the reverse side of the sheet.
\end{pbsect}
%
\moveexpembedded{four}{Director}
\brln
%
\textit{You get these two:}\\
%
\begin{move*}{I'm Here For A Reason}
There’s something The Board expects you to do.  Work out the details with the Keeper, based on your duties as director of the bureau.  You cannot die until it comes to pass.  If you \textbf{die in play}, then you must spend a luck point.  You will then, somehow, recover or be returned to life.  Once your task is done, or you use up all your luck, all bets are off.
\end{move*}%
\br[1mm]
%
\begin{move*}{Advisors}
When you consult your heads of staff about an ongoing situation, take +1 to \textbf{investigate a mystery}.
\end{move*}%
\br[3mm]
%
\textit{Then pick two of these:}\\
%
\begin{move}{The Big Entrance}
When you \textbf{make a commanding entrance into a situation}, roll +Cool.  On a 10+ everyone stops to watch and listen until you finish your opening speech.  On a 7-9 you pick one person or entity to stop, watch and listen until you finish talking.  On a miss, you’re marked as the biggest threat by all enemies who are present.
\end{move}%
\br[1mm]
%
\begin{move}{Trust Your Gut}
When you \textbf{consult your instincts} about what to do next, roll +Weird: On a 10 or more, the Keeper will tell where you should go. Wherever that is, it will be important. You get +1 ongoing on the way to this place. On a 7-9, the Keeper will tell you a general direction to go. Take +1 forward as you explore that. On a miss, your instincts lead you into danger.
\end{move}%
%
}{% -- pg.1/2 col.3/3
%
\begin{move}[]{Dutiful}
When your requirements as director of the bureau rear their ugly head, and you \textbf{act in accordance with your Duties} (either Bureau or Board), then mark experience and take +1 forward.
\end{move}%
\br[1mm]
%
\begin{move}{Resilience}
You heal faster than normal people.  Any time your harm gets healed, heal an extra point.  Additionally, your wounds count as 1-harm less for the purpose of the Keeper’s harm moves.
\end{move}%
\br[1mm]
%
\begin{move}{Hotline}
When you \textbf{make a call on the Hotline}, tell the Keeper whether you’re trying to reach the Board, a different astral entity, or hear the echoing thoughts of the recently deceased, and roll +Sharp.  On a 10+, they tell you something useful.  On a 7-9, they tell you something interesting, but it’s up to you to make it useful.
\end{move}
%
\begin{pbsect}{GEAR}
You can have protective gear worth 1-armour, if you want.  You have the service weapon, like any Director would.
\end{pbsect}%
\brln
%
\begin{pbsect}{THE SERVICE WEAPON}
Design your weapon by choosing a base, three forms, and a material.  Your Service Weapon always counts as a weakness against the entities you fight.
\end{pbsect}%
\br[2mm]
%
\optionsparen{Base}{choose 1}%
{Hilt (2-harm hand balanced),
Handgun (2-harm close reload)}%
\br[2mm]
%
\optionsblurb{Forms}%
{Take Grip and choose 2 more.  Only one form is active at a time.  The service weapon can instantly reconfigure from one form to another any time you are in physical contact with it.}%
    {Grip (+1 harm),
    Pierce (Ignore-armor),
    Spin (Messy),
    Shatter (Area),
    {Surge (Thrown, Explosive, Loud)}}%
%
}\pagebreak% -- back page
%
\playbookpage{%
%
% -- pg.2/2 col.1/3
%
\begin{blurbparen}{Material}{pick one}
{Finally, pick what material the service weapon is made from.  “steel”, “cold iron”, “silver”, “wood”, “stone”, “bone”, “teeth”, “obsidian”, or anything else you want.}
\end{blurbparen}
%
\begin{pbsect}{Your Duties as Director}
\optionsparen{How You Were Selected}{pick one}%
    {Found The Service Weapon,
    The Bureau Found You,
    Survived An AWE,
    The Last Director Trained You,
    Sought Out By An Astral Entity}
\brln
Then pick two Bureau and two Board tags for your duty to the agency from the lists below.  This is how The Bureau’s future will unfold.  It’s okay to pick contradictory tags: that means there is management strife between The Board and The Bureau.
\br[2mm]
When you \textbf{mark off a point of luck}, the Keeper will throw something from your duties at you.
\end{pbsect}%
\br[4mm]
%
\optionstwocolparen{The Bureau}{pick two}%
    {Things are better with you in charge,
    A favored inner circle,
    A bountiful threshold,
    A normal life,
    The Bureau before the Board,
    Succumb to the power,
    Unethical research,
    You must save everyone,
    Employee dissent,
    You can save the Bureau}
%
\optionstwocolparen{The Board}{pick two}%
    {Conspiracy,
    Your employees safety doesn’t matter,
    Enraged Altered Items,
    Sacrifice,
    The Board runs the Bureau,
    The source of the Paranormal,
    Obedience,
    Dissent in the astral plane,
    Nearing your retirement,
    No normal life}%
%
}{% -- pg.2/2 col.2/3
%
\optionsparen{Ratings}{pick one line}%
    {{Charm+2, Cool-1, Sharp+1, Tough+2, Weird-1},
    {Charm-1, Cool+2, Sharp+1, Tough+2, Weird-1},
    {Charm+1, Cool+2, Sharp+1, Tough+1, Weird-1},
    {Charm-1, Cool+1, Sharp+2, Tough-1, Weird+2},
    {Charm+1, Cool+2, Sharp-1, Tough-1, Weird+2}}
%
\introductions{Director}%
\brln
%
\begin{history}
\begin{itemize}
\item You are close blood relations.  Ask them exactly how close.
\item They are destined to be your replacement.  Tell them how this was revealed.
\item You worked your way up through the Bureau together, and trust them totally.
\item Romantically entangled.  Or fated to be romantically entangled.
\item A rival at first, but you came to a working arrangement.
\item They could have been the Director instead of you.  Tell them how they failed.
\item You saved their life, back before they knew the paranormal was real.  Tell them what you saved them from.
\end{itemize}
\end{history}
%
}{% -- pg.2/2 col.3/3
%
\levelingup
%
\improvementsonecol{%
    {Get +1 Charm, max +3.},
    {Get +1 Weird, max +3.},
    {Get +1 Cool, max +3.},
    {Get +1 Sharp, max +3.},
    {Get +1 Tough, max +3.},
    {Take another Director move.},
    {Take another Director move.},
    {Take a move from another playbook.},
    {Take a move from another playbook.},
    {Take another Service Weapon Form.}
}{%
    {Get +1 to any rating, max +3.},
    {Get back one used Luck point.},
    {Create a second Bureau Agent to play as well as this one.},
    {Mark two of the basic moves as advanced.},
    {Mark another two of the basic moves as advanced.},
    {Step down as Director: forfeit your Service Weapon and pick a new playbook.},
    {Appoint a new Director, give them the Service Weapon, and retire.}
}%
%
}%
% -- end playbook
\pagebreak

% The Investigator ---------------------------------------
% the investigator ---------------------------
%
% -- front page
%
\playbookpage{%
%
% -- pg.1/2 col.1/3
%
\pbcommon{Investigator}%
{Someone sends a handwritten letter about a stapler that growls and bites the other office supplies and I'm the one getting flown out to Minnesota to check it out.}%
{When you mark off a point of luck, the Keeper will haul out the department's red tape.}%
%
}{% -- pg.1/2 col.2/3
%
\moveexp{three}{Investigator}
\brln
%
\begin{move}{Connect the Dots}
At the beginning of each mystery, if you \textbf{look for the wider patterns} that current events might be part of, roll +Sharp.  On a 10+ hold 3, and on a 7-9 hold 1.  Spend your hold during the mystery to ask the keeper any one of the following questions:
\holdoptions%
    {Is this person connected to current events more than they are saying?,
    When and where will the next critical event occur?,
    What caused this paranormal occurrence?,
    Is this connected to previous mysteries we have investigated?,
    How does this mystery connect to the bigger picture?}%
\end{move}%
\br[1mm]
%
\begin{move}{Ockham’s Exposition}
When you \textbf{first encounter something strange}, you may ask the Keeper what sort of thing it is.  They will tell you if the cause is natural, a person, or the paranormal.  You gain +1 forward when dealing with it.
\end{move}%
\br[1mm]
%
\begin{move}{Suspicous Mind}
When you think someone is lying to you, tell the Keeper or that Agent, whomever plays the character.  They must tell you honestly if a lie was spoken, but are not required to point out the lie, nor are they required to tell you the truth.
\end{move}%
\br[1mm]
%
\begin{move}{Government Friends}
You know a lot of people in government institutions.  When you contact a \textbf{government friend to help} you with a mystery, roll +Charm.  On a 10+ they’re available and helpful - they can cover something up, provide clearance, get you special information, or provide protection.  On a 7-9 they’re prepared to help, but it’s either going to take some time or you’re going to have to do part of it yourself.  On a miss, you burn some bridges.
\end{move}%
%
}{% -- pg.1/2 col.3/3
%
\begin{move}[]{See, it all fits together}
You can use Sharp instead of Charm when you \textbf{manipulate someone}.
\end{move}%
\br[1mm]
%
\begin{move}{The Things I've Seen}
When you \textbf{encounter a paranormal entity or phenomenon}, you may declare that you have seen it before.  The keeper may ask you some questions about that encounter, and will then tell you one useful fact you learned and one danger you need to watch out for (maybe right now).
\end{move}%
\br[1mm]
%
\begin{move}{"Just One More Thing"}
When you \textbf{ask a suspect leading questions}, roll +Charm. On a 10+ hold 2, on a 7-9 hold 1, on a miss hold 1 but something bad is going to happen too. Spend your hold to ask questions from this list:
\holdoptions%
{Ask one question from the investigate a mystery list.,
Are they in control of themselves?,
What is something you left out that you didn’t want me to notice?,
Are you complicit with any ongoing paranormal activity?,
Did you have something to do with this event?}
\end{move}
%
}\pagebreak% -- back page
%
\playbookpage{%
%
% -- pg.2/2 col.1/3
%
\begin{pbsect}{GEAR}[]
You get two self defense items, and two investigation tools.
\end{pbsect}
\br[2mm]
%
\optionsparen{Self Defense Items}{choose two}%
    {Walking stick (1-harm hand innocuous),
    Small handgun (2-harm close reload loud),
    Small knife (1-harm hand messy),
    Martial arts training (1-harm hand innocuous),
    Incapacitating spray (0-harm hand irritating),
    Heavy flashlight (1-harm hand innocuous)}
\brln
%
\optionsparen{Investigation Tools}{choose two}%
    {Dowsing Rods,
    A CB Hand Radio attuned to only picks up paranormal wavelengths,
    {Black rock containment gear (mittens, tongs, a box with a locking lid)},
    Bag of film cameras and microphones,
    Paranormal measuring tools}%
%
\begin{pbsect}{DEPARTMENT RESOURCES}
The investigations department itself comes with its own perks and red tape.  Pick two perks that the department offers you while you're on the job, and a red tape issue you have to battle through to get the job done
\end{pbsect}
\br[2mm]
%
\optionsparen{Department Perks}{choose two}%
    {Well-financed,
    Rigorous training,
    Cover identities,
    Offices all over the place,
    Good intel,
    Recognized Authority}
\br[1pt]
%
\optionsparen{Department Red Tape}{choose one}%
    {Bureaucratic,
    Hostile superiors,
    Interdepartmental rivalry,
    Live capture policy,
    Trainee dumping ground}
%
}{% -- pg.2/2 col.2/3
%
\optionsparen{Ratings}{pick one line}%
    {{Charm+1, Cool+1, Sharp+2, Tough-1, Weird=0},
    {Charm=0, Cool+1, Sharp+2, Tough-1, Weird+1},
    {Charm+1, Cool-1, Sharp+2, Tough+1, Weird=0},
    {Charm+1, Cool-1, Sharp+2, Tough=0, Weird+1},
    {Charm-1, Cool-1, Sharp+2, Tough=0, Weird+2}}
%
\introductions{Investigator}%
\brln
%
\history%
    {{They’re somehow tied into it all. You’ve been keeping an eye on them.},
    {They’re a close relative. Ask them to decide exactly what.},
    {Old friends, who originally met through a long chain of coincidences.},
    {You're both members of the same support group.},
    {The signs all pointed to working together. So you found them and now you work together.},
    {They were involved in a supernatural event similar to your first encounter. Perhaps it was the same event, or perhaps you investigated their event later. Ask them how the event affected them.},
    {You met when you were each investigating separate mysteries. Tell them what trick you used to protect them from weirdness and ask them how they saved you from a danger.}}%
%
}{% -- pg.2/2 col.3/3
%
\levelingup
%
\improvementsonecol{%
    {Get +1 Charm, max +3.},
    {Get +1 Weird, max +3.},
    {Get +1 Cool, max +3.},
    {Get +1 Sharp, max +3.},
    {Take another Investigator move.},
    {Take another Investigator move.},
    {Take a move from another playbook.},
    {Get another resource tag for your agency or change a red tape tag.},
    {The Department assigns you a partner.  Use the Ranger’s Squad move to tag and describe them.}
}{%
    {Get +1 to any rating, max +3.},
    {Get back one used Luck point.},
    {Create a second Bureau Agent to play as well as this one.},
    {Mark two of the basic moves as advanced.},
    {Mark another two of the basic moves as advanced.},
    {Jump departments: change to a new playbook.},
    {Get promoted to the Head of Investigations.  This character becomes an NPC.  Start a new character.},
    {Retire to safety.}
}%
%
}%
% -- end playbook
\pagebreak

% The Ranger ---------------------------------------
% the ranger ---------------------------
%
% -- front page
%
\playbookpage{%
%
% -- pg.1/2 col.1/3
%
\pbcommon{Ranger}%
{Bureau leadership always goes on about how vital the secure and contain protocol is, how it's our duty to protect these things.  Bullshit.  I protect us.}%
{When you spend a point of luck, the Keeper will bring about some danger or complication to one of your squad members.}%
%
}{% -- pg.1/2 col.2/3
%
\moveexp{four}{Ranger}
\br[1mm]
%
\textit{Pick three of these:}\\
%
\begin{move}{Bottle It Up}
If you want, you can take up to +3 bonus when you \textbf{act under pressure}. For each +1 you use, the Keeper holds 1. That hold can be spent later—one for one—to give you -1 on any move except act under pressure.
\end{move}%
\br[1mm]
%
\begin{move}{Unfazeable}
Take +1 Cool (Max +3).
\end{move}%
\br[1mm]
%
\begin{move}{Battlefield Awareness}
You always know what’s happening around you, and what to watch out for. Take +1 armour (max 2-armour) on top of whatever you get from your gear.
\end{move}%
\br[1mm]
%
\begin{move}{Leave No One Behind}
In combat, when you \textbf{help someone escape}, roll +Sharp. On a 10+ you get them out clean. On a 7-9, you can either get them out or suffer no harm, you choose. On a miss, you fail to get them out and you’ve attracted hostile attention.
\end{move}%
\br[1mm]
%
\begin{move}{Tactical Genius}
When you \textbf{read a bad situation}, you may roll +Cool instead of +Sharp.
\end{move}%
\br[1mm]
%
\begin{move}{Medic}
You have a full first aid kit, and the training to heal people. When \textbf{you do first aid}, roll +Cool. On a 10+ the patient is stabilized and healed of 2 harm. On a 7-9 choose one: heal 2 harm or stabilize the injury. On a miss, you cause an extra 1 harm. This move takes the place of regular first aid.
\end{move}%
\br[1mm]
%
\begin{move}{Trust Me}
When you \textbf{tell a normal person the truth in order to protect them from danger}, roll +Charm. On a 10+ they’ll do what you say they should, no questions asked. On a 7-9 they do it, but the Keeper chooses one from:
\holdoptions%
    {They ask you a hard question first.,
    They stall and dither a while.,
    They have a “better” idea.}%
    On a miss, they’re going to think you’re crazy and probably dangerous too.
\end{move}
%
}{% -- pg.1/2 col.3/3
%
\textit{And you get this one:}\\
%
\begin{move*}{Call For Backup}
When you \textbf{deal with the Bureau}, requesting help or gear, or making excuses for a failure, roll +Sharp. On a 10+, you’re good—your request for gear or personnel is okayed, or your slip-up goes unnoticed. On a 7-9, things aren’t so great. You might get chewed out by your superiors and there’ll be fallout, but you get what you need for the job. On a miss, you screwed up: you might be suspended or under investigation, or just in the doghouse. You certainly aren’t going to get any help until you sort it all out.
\end{move*}
%
\begin{pbsect}{GEAR}
Pick one heavy weapon, two normal weapons, a vehicle, and describe your squad.  You get either a flak vest (1-armour hidden) or combat armor (2-armour heavy) for protection.
\end{pbsect}
\br[2mm]
%
\optionsparen{Heavy Weapons}{choose one}%
    {Assault rifle (3-harm far area loud reload),
    Grenade launcher (4-harm far area messy loud reload),
    Sniper rifle (4-harm far),
    Grenades (4-harm close area messy loud),
    Submachine gun (3-harm close area loud reload)}
\brln
%
\optionsparen{Normal Weapons}{choose two}%
    {.38 revolver (2-harm close reload loud)1,
    9mm (2-harm close loud),
    Hunting rifle (2-harm far loud),
    Shotgun (3-harm close messy),
    Big knife (1-harm hand)}%
%
}\pagebreak% -- back page
%
\playbookpage{%
%
% -- pg.2/2 col.1/3
%
\begin{pbsect}{YOUR SQUAD}[]
You don’t work alone, you’re one part of a highly trained team of operatives.
\end{pbsect}
%
\begin{pbsect}{VEHICLE}
The bureau provides your squad with an outfitted truck, van, or car built for handling AWEs. Choose two good things and one bad thing about it.
\br[2mm]
\textit{Good things}: roomy; surveillance gear; fast; stealthy; intimidating; classic; medical kit; sleeping space; toolkit; concealed weapons; anonymous; armoured (+1 armour inside); tough; paranormal item container.
\br[2mm]
\textit{Bad things}: loud; obvious; temperamental; beaten-up; gas-guzzler; uncomfortable; slow; old.
\end{pbsect}
%
\begin{pbsect}{SQUAD MEMBERS}
Your squad consists of two, three, or four members (including yourself).  Name each member, give them a specialty and a flaw from the list, and write a sentence about your relationship.
\br[2mm]
\textit{Specialties}: driver; medic; strategy and tactics; sniper; demolitions; paranormal containment; psychic; veteran; hacker; stealthy; heavy weapons.
\br[2mm]
\textit{Flaws}: green; aggro; callous; bossy; paranoid; compromised; addict; strict; short-sighted; bad aim.
\end{pbsect}%
\brln
%
\begin{blurb}{Squad Member 1:}
\vspace{3pt}
Specialty: \rule{0.3\linewidth}{0.5pt} Flaw: \rule{0.3\linewidth}{0.5pt}
\br[1mm]
Relationship: \rule{0.675\linewidth}{0.5pt}
\br[1mm]
Tags: \rule{0.825\linewidth}{0.5pt}
\end{blurb}%
\br[4mm]
%
\begin{blurb}{Squad Member 2:}
\vspace{3pt}
Specialty: \rule{0.3\linewidth}{0.5pt} Flaw: \rule{0.3\linewidth}{0.5pt}
\br[1mm]
Relationship: \rule{0.675\linewidth}{0.5pt}
\br[1mm]
Tags: \rule{0.825\linewidth}{0.5pt}
\end{blurb}%
\br[4mm]
%
\begin{blurb}{Squad Member 3:}
\vspace{3pt}
Specialty: \rule{0.3\linewidth}{0.5pt} Flaw: \rule{0.3\linewidth}{0.5pt}
\br[1mm]
Relationship: \rule{0.675\linewidth}{0.5pt}
\br[1mm]
Tags: \rule{0.825\linewidth}{0.5pt}
\end{blurb}
}{% -- pg.2/2 col.2/3
%
\optionsparen{Ratings}{pick one line}%
    {{Charm=0, Cool+2, Sharp-1, Tough+2, Weird-1},
    {Charm-1, Cool+2, Sharp+1, Tough+1, Weird=0},
    {Charm+1, Cool+2, Sharp+1, Tough-1, Weird=0},
    {Charm-1, Cool+2, Sharp+1, Tough=0, Weird+1},
    {Charm=0, Cool+2, Sharp+2, Tough-1, Weird-1}}
%
\introductions{Ranger}%
\brln
%
\history%
    {{You went through hell together: maybe an AWE, maybe military service, maybe time in an institution. Whatever it was, it bound you together, and you have total trust in each other.},
    {Your relationship with them has romantic potential. So far it hasn’t gone further.},
    {They’re on the Bureau’s internal watch list, and you’ve been keeping an eye on them.},
    {You are related. Tell them how close.},
    {You met on a mission and worked together unofficially. And successfully.},
    {You were friends back in training, before the Bureau recruited you. This could be military, law enforcement, or some weirder school: decide the details between you.},
    {They pulled you (and maybe your team) out of a terrible FUBARed mission.},
    {You got sent to “deal with them” as a hazard to the Bureau’s policies one time. Tell them how you resolved this.}}%
%
}{% -- pg.2/2 col.3/3
%
\levelingup
%
\improvementsonecol{%
    {Get +1 Charm, max +3.},
    {Get +1 Cool, max +3.},
    {Get +1 Sharp, max +3.},
    {Get +1 Tough, max +3.},
    {Take another Ranger move.},
    {Take another Ranger move.},
    {Take a move from another playbook.},
    {Take a move from another playbook.},
    {Pick another Serious Weapon.},
    {Add another good tag to your vehicle or change the bad tag.},
    {Add someone to your squad.}
}{%
    {Get +1 to any rating, max +3.},
    {Get back one used Luck point.},
    {Pick a squadmate as a second Bureau Agent to play as well as this one.},
    {Mark two of the basic moves as advanced.},
    {Mark another two of the basic moves as advanced.},
    {Jump departments: change to a new playbook.},
    {Get promoted to the Head of Operations.  This character becomes an NPC.  Start a new character.},
    {Retire to safety.}
}%
%
}%
% -- end playbook
\pagebreak

% The Janitor ---------------------------------------
%%%%%%%%%%%%%%%%%%%%%%%%%%%%%%%%%%%%%%%%%%%%%%%%%%%%%%%%%%%%%%%%%%%%%%%%%%%%%%%%%%%%%%%%
%
% Altered World Event of the Week is a Monster of the Week derivative.
%
% Much of the textual content in the playbooks is copied from the original playbooks,
% following approval by Michael Sands for the free distribution of the AWEotW document.
% See: https://genericgames.co.nz/third_party_policy/ for his third party policy.
%
% Re-use of this script is for personal reproduction of the formatting and generation
% of other custom playbooks.  Any reproduction of the text should seek approval per the
% aforementioned generic games third party policy.
%
% Monster of the Week is copyrighted by Evil Hat Productions, LLC and Generic Games.
%
%%%%%%%%%%%%%%%%%%%%%%%%%%%%%%%%%%%%%%%%%%%%%%%%%%%%%%%%%%%%%%%%%%%%%%%%%%%%%%%%%%%%%%%%

% the janitor ---------------------------
%
% -- front page
%
\playbookpage{%
%
% -- pg.1/2 col.1/3
%
\pbcommon{Janitor}%
{Someone called looking for an assistant.  I answered the call.  They let me into the building, showed me around the place.  I've been here ever since.}%
{When you spend a point of luck, hold 1.  When you spend this hold, the oceanview motel lightswitch cord appears somewhere nearby, say where it is.}%
%
}{% -- pg.1/2 col.2/3
%
\moveexp{four}{Janitor}%
\br[2mm]
%
\textit{Pick two of the following three.  You cannot pick the third when you level up.  Only the Janitor can pick these moves.}
%
\br[1mm]
%
\begin{move}{Supply Closets}
When you’re inside the Bureau, or another building you’ve previously worked within, and need something that you could conceivably find in your janitorial supplies, there is a nearby supply closet that has it.
\br[2mm]
If you \textbf{need something that isn’t normally found in janitorial supplies}, open the nearest supply closet and roll +Weird.  On a 10+ it’s there, just like you needed.  On a miss, there’s something in the closet you aren’t going to like.  On a 7-9, both.
\end{move}%
\br[2mm]
%
\begin{move}{The Janitor Always Has The Keys}
When you need access to a part of the Bureau, or another building you’ve previously worked within, you either already have a set of keys that unlock the doors, or you know an alternate way through the building to get to the same spot, the Keeper will tell you which one.
\end{move}%
\br[2mm]
%
\begin{move}{Always Paying Attention}
Take two moves from any unused playbook.
\end{move}
%
}{% -- pg.1/2 col.3/3
%
\textit{Then pick two of these:}\\
%
\begin{move}{Oops!}
If you want to \textbf{stumble across something important}, tell the Keeper. You will find something important and useful, although not necessarily related to your immediate problems.
\end{move}%
\br[1mm]
%
\begin{move}{Let's Get Out of Here!}
If you can \textbf{protect someone} by telling them what to do, or by leading them out, roll +Charm instead of +Tough.
\end{move}%
\br[1mm]
%
\begin{move}{Through the Air Ducts}
When \textbf{you need to escape}, name the route you’ll try and roll +Sharp. On a 10+ you’re out of danger, no problem. On a 7-9 you can go or stay, but if you go it’s going to cost you (you leave some-thing behind or something comes with you). On a miss, you are caught halfway out.
\end{move}%
\br[1mm]
%
\begin{move}{The Power of Heart}
When fighting a monster, if you \textbf{help someone}, don’t roll +Cool. You automatically help as though you’d rolled a 10.
\end{move}%
\br[1mm]
%
\begin{move}{Don't Worry, I'll Check It Out}
Whenever you go off by yourself to check out somewhere (or something) scary or dangerous, mark experience.
\end{move}%
\br[1mm]
%
\begin{move}{So Much Work To Do}
While inside the oldest house, you heal faster than normal people. Any time your harm gets healed, heal an extra point. You are immune to all the harm move effects under ‘0-harm’ and ‘1-harm’ (when the Keeper would apply these, you ignore it).
\end{move}%
\br[1mm]
%
\begin{move}{Just Another Day}
When you have to \textbf{act under pressure} due to an altered item, phenomenon, or paranormal effect, you may roll +Weird instead of +Cool.
\end{move}%
\br[1mm]
%
}\pagebreak% -- back page
%
\playbookpage{%
%
% -- pg.2/2 col.1/3
%
\begin{pbsect}{TALISMAN}[]
You have an Object of Power that helps or protects you. Define it, and its power, with the Keeper’s agreement. The object is one of: a janitorial supply item, a discrete and old piece of electronics, or something that plays music.  The Bureau is not aware that this is an object of power, or that you have objects of power in your possession.
\end{pbsect}
\br[2mm]
%
\begin{pbsect}{HOUSEKEEPER}
You may not always know what the people in the Bureau are doing, or why they’re doing it, but no one knows the oldest house itself like you do.  Whenever you ask the Keeper where a place or thing is located inside the house, even if that location is supposed to be secret, even if the house has shifted, they will tell you where it is (or is supposed to be).
%
\begin{center}
Knowledge: \checkbox[7]
\end{center}
%
At any time during play you may mark a box of Knowledge to do one of the following:
\holdoptions
    {{Describe a sector or threshold within the oldest house.  That place now exists.  With the Keeper’s agreement, describe details such as whether it is overt or secret, in use or vacated, and what can be found there.},
    {Ask the Keeper about any paranormal thing that exists inside the oldest house.  They will tell you three things about that item.},
    {Ask or tell the Keeper about a ritual or procedure employees use to control or work within the oldest house.  You now know as much as any other bureau employee about it, and are able to reproduce the effects.}}
\end{pbsect}%
%
}{% -- pg.2/2 col.2/3
%
\optionsparen{Ratings}{pick one line}%
    {{Charm+2, Cool+1, Sharp=0, Tough+1, Weird-1},
    {Charm+2, Cool-1, Sharp+1, Tough+1, Weird=0},
    {Charm+2, Cool=0, Sharp-1, Tough+1, Weird+1},
    {Charm+2, Cool=0, Sharp+1, Tough+1, Weird-1},
    {Charm+2, Cool+1, Sharp+1, Tough=0, Weird-1}}
%
\introductions{Janitor}
%
\history%
    {{You are close relations. Tell them exactly how you’re related.},
    {You helped get them their job in the Bureau.  Tell them what strings you pulled.},
    {Workplace buddies.  Your responsibilities caused you to bump into each other frequently and that developed into a lasting friendship.},
    {They’ve been breaking the Bureau’s rules and they think they’re getting away with it.  But you’ve seen them do it.  Ask them what you’ve seen them do.},
    {You’re a bit suspicious, or extra welcoming, of them (maybe due to their unnatural powers or something like that).},
    {You accidentally ruined some of their work while cleaning up their space.  Ask them what happened.},
    {Due to an unlikely chain of events you saved their life from a safety breach within the Bureau. Tell them how.}}%
%
}{% -- pg.2/2 col.3/3
%
\levelingup
%
\improvementsonecol{%
    {Get +1 Charm, max +3.},
    {Get +1 Cool, max +3.},
    {Get +1 Weird, max +3.},
    {Get +1 Tough, max +3.},
    {Take another Janitor move.},
    {Take another Janitor move.},
    {Take a move from another playbook.},
    {Take a move from another playbook.},
    {Take a move from another playbook.},
    {Get back one used Knowledge or Luck point.},
    {Get back one used Knowledge or Luck point.}
}{%
    {Get +1 to any rating, max +3.},
    {Get back one used Knowledge and Luck point.},
    {Get back one used Knowledge and Luck point.},
    {Mark two of the basic moves as advanced.},
    {Mark another two of the basic moves as advanced.}
}%
%
}%
% -- end playbook
\pagebreak

% The Researcher ---------------------------------------
% the researcher ---------------------------
%
% -- front page
%
\playbookpage{%
%
% -- pg.1/2 col.1/3
%
\pbcommon{Researcher}%
{Tell me, given the choice which option would you pick: a lab full of human blood samples, or a lab full of interdimensional beings?}%
{When you spend a point of luck, something inside your lab goes awry or missing.}%
%
}{% -- pg.1/2 col.2/3
%
\moveexp{two}{Researcher}%
%
\begin{move}{I've Read About This Sort of Thing}
Roll +Sharp instead of +Cool when you \textbf{act under pressure}.
\end{move}%
\br[1mm]
%
\begin{move}{Often Right}
When an \textbf{agent comes to you for advice} about a problem, give them your honest opinion and advice. If they take your advice, they get +1 ongoing while following your advice, and you mark experience.
\end{move}%
\br[1mm]
%
\begin{move}{Preparedness}
When you \textbf{need something unusual or rare}, roll +Sharp. On a 10+, you have it here right now. On a 7-9 you have it, but not here: it will take some time to get it. On a miss, you know where it is, but it’s somewhere real bad.
\end{move}%
\br[1mm]
%
\begin{move}{The Person With The Plan}
\textbf{At the beginning of each mystery}, roll +Sharp. On a 10+ hold 2, on a 7-9 hold 1. Spend the hold to be where you need to be, prepared and ready. On a miss, the Keeper holds 1 they can spend to put you in the worst place, unprepared and unready.
\end{move}%
\br[1mm]
%
\begin{move}{Dark Past}
If \textbf{you trawl through your memories} for something relevant to the case at hand, roll +Sharp. On a 10+ ask the Keeper two questions from the list below. On a 7-9 ask one. On a miss, you can ask a question anyway but that will mean you were personally complicit in creating the situation you are now dealing with. The questions are:
\holdoptions%
    {{When I last dealt with this paranormal event (or one of its kind), what did I learn?},
    {What research do I know that could help here?},
    {Do I know anyone who might be behind this?},
    {Who do I know who can help us right now?}}
\end{move}%
%
}{% -- pg.1/2 col.3/3
%
\begin{move}{Ritual Experiments}
When you \textbf{perform an experiment on an altered item} you can treat that act as something the item needs or wants.  Roll +Tough.  On a 10+ the item is subdued for a time.  On a 7-9 the item reacts negatively to the experiment, but you’re ready for it.  On a miss, that wasn’t at all what the item wanted, and now you’re stuck with it and its erratic behavior.
\end{move}%
\br[1mm]
%
\begin{move}{Helping Hands}
You have a lab assistant whose job is to help you out with whatever you need.  When \textbf{you put your assistant in danger} in the name of science, roll +Charm.  10+ they’re fine, and besides this is how real science happens, they should get used to it.  7-9 pick one:
\holdoptions%
    {{They get hurt in the process.},
    {They complain to someone important about what they’re going through.},
    {Whatever you’re researching gets damaged or set loose.}}%
On a miss, prepare for the worst...
\end{move}%
\br[1mm]
%
\begin{move}{Work From Anywhere}
You have a tiny, mobile lab built into a van, moving truck, trailer, or shipping container.  Pick one of your three lab options.  That part of your lab is accessible wherever you are, no matter where you are, provided that the vehicle which carries it can make it there.
\end{move}%
\br[1mm]
%
%
}\pagebreak% -- back page
%
\playbookpage{%
%
% -- pg.2/2 col.1/3
%
\begin{pbsect}{RESEARCH LAB}[]
You have a personal lab, a safe place to work, inside the oldest house. Pick three of the options below for your lab:
\end{pbsect}
\br[2mm]
%
\textbf{Published Research}: You have access to extensive publications on paranormal events.  When you spend some time studying the literature, take +1 forward to investigate the mystery.
\br[2mm]
\textbf{Ritual Research}: You have access to extensive documentation on previously successful rituals.  When you refer to these documents, describe a past event where a similar paranormal event, item, or entity was successfully contained, then take +1 forward on Ritual or Fringe Science.
\br[2mm]
\textbf{Reinforced Container}: Your lab has a reinforced room that can contain destructive paranormal items and entities without too much worry of them breaking out.
\br[2mm]
\textbf{Infirmary}: You can relieve people of paranormal influence, and have the space for one or two to recuperate.  The Keeper will tell you how long any patient’s recovery is likely to take, and if you need extra supplies or help.
\br[2mm]
\textbf{Workshop}: You have a space for building and repairing gadgets and paranormal equipment.  Work out with the keeper how long any repair or construction will take, and if you need extra supplies and help.
\br[2mm]
\textbf{Black Rock Walls}: This room is isolated from every kind of paranormal influence that you know about. Anything you stash in there can’t be found, can’t affect anything outside the lab, and can’t get out without some destruction.
\br[2mm]
\textbf{Safe Room}: Your lab comes with a safe room that is protected by normal and paranormal means.  You and up to two others can hide out there for a few days, safe from pretty much everything.
\br[2mm]
\textbf{Ritual Laboratory}: Your lab is equipped with all kinds of weird tools and supplies useful for doing rituals (like the move Ritual or Grand Ritual).
%
}{% -- pg.2/2 col.2/3
%
\optionsparen{Ratings}{pick one line}%
    {{Charm-1, Cool+1, Sharp+2, Tough+1, Weird=0},
    {Charm=0, Cool+1, Sharp+2, Tough-1, Weird+1},
    {Charm+1, Cool-1, Sharp+2, Tough+1, Weird=0},
    {Charm-1, Cool+1, Sharp+2, Tough=0, Weird+1},
    {Charm-1, Cool=0, Sharp+2, Tough-1, Weird+2}}
%
\introductions{Researcher}
%
\history%
{{They are your student, apprentice, assistant, ward, or child. Between you, decide which.},
{They came to you for advice, and your advice got them out of trouble. Ask them what the trouble was.},
{They know about some of your unethical research, but they’ve agreed to keep quiet about them. Tell them what they know.},
{A distant relation. Tell them exactly what.},
{You were previously both members of an eldritch group, now disbanded. Ask them why they left, then tell them why you did.},
{They once helped you get a singular item that is now in your lab. Tell them what it was.},
{You both had to deal with the former head of research, who went crazy or whose obsessions got too extreme. Ask them how it ended.},
{You saved their life in a tight spot. Tell them what happened.}}%
%
}{% -- pg.2/2 col.3/3
%
\levelingup
%
\improvementsonecol{%
    {Get +1 Charm, max +3.},
    {Get +1 Weird, max +3.},
    {Get +1 Cool, max +3.},
    {Get +1 Sharp, max +3.},
    {Take another Researcher move.},
    {Take another Researcher move.},
    {Add an option to your Lab.},
    {Add an option to your Lab, or add a second option to your Remote Lab.},
    {Take a move from another playbook.},
    {Take a move from another playbook.}
}{%
    {Get +1 to any rating, max +3.},
    {Get back one used Luck point.},
    {Create a second Bureau Agent to play as well as this one.},
    {Mark two of the basic moves as advanced.},
    {Mark another two of the basic moves as advanced.},
    {Jump departments: change to a new playbook.},
    {Get promoted to the Head of Research.  This character becomes an NPC.  Start a new character.},
    {Retire to safety.}
}%
%
}%
% -- end playbook
\pagebreak

% The Outsider ---------------------------------------
% the outsider ---------------------------
%
% -- front page
%
\playbookpage{%
%
% -- pg.1/2 col.1/3
%
\pbcommon{Outsider}%
{You spend your life watching a movie, thinking that's all there is.  You only realize there's more to find when it reaches through the screen to grab you.}%
{When you spend a point of Luck, your first encounter comes up in play. It could be a flashback, new occurrence, or related event.}%
%
}{% -- pg.1/2 col.2/3
%
\moveexp{three}{Outsider}
\brln
%
\textit{You get this one:}
%
\br[2mm]
\begin{minipage}[t]{0.05\linewidth}
\checkedbox
\end{minipage}
%
\begin{minipage}[t]{0.94\linewidth}
\textbf{\normalsize First Encounter:}
One strange event started you down this path, sparking your need to discover the truth behind the unexplained. Decide what that event was: pick a category below and take the associated move. Then tell everyone what happened to you (or someone close to you).%
\br[1mm]
%
\begin{move}{AWE Survivor}
You take note of any reports of paranormal behavior. Whenever you first see a new type of AWE, you may immediately ask one of the \textbf{investigate a mystery} questions.
\end{move}%
\br[1mm]
%
\begin{move}{Non-Euclidean}
Things are not fixed. You never need \textbf{act under pressure} when supernatural forces alter the environment around you, and you get 2-armour against harm from sudden changes to the laws of physics
\end{move}%
\br[1mm]
%
\begin{move}{Astral Friend}
Something in the astral plane is looking out for you.  You start with 1 extra Luck: \checkbox
\end{move}%
\br[1mm]
%
\begin{move}{Strange Dangers}
You are always watching for hazards. When you have no armour, you still count as having 1-armour
\end{move}%
\br[1mm]
%
\begin{move}{Cosmic Insight}
You have peeked beyond the veil of this plane. You never need to \textbf{act under pressure} due to feelings of paranoia, despair, or isolation.
\end{move}%
\br[1mm]
%
\begin{move}{Bureau Interference}
Long ago, the Bureau took someone away who was special to you.  You’ve been trying to find them ever since.  Whenever \textbf{you directly confront someone} about them hiding information or evidence from you, they must do one of the following: give you the truth, bail out of the situation, or find a way to stop you from asking so damn many nosy questions, quick.  
\end{move}
%
\end{minipage}
%
}{% -- pg.1/2 col.3/3
%
\textit{Then pick two of these:}
%
\begin{move}{NEVER AGAIN}
In a dangerous situation, you may choose to \textbf{protect someone} without rolling, as if you had rolled a 10+, but you may not choose to “suffer little harm.”
\end{move}%
\br[1mm]
%
\begin{move}{DIY Surgery}
When you \textbf{do quick and dirty first aid} on someone (including yourself), roll +Cool. On a 10+ it’s all good, it counts as normal first aid, plus stabilize the injury and heal 1 harm. On a 7-9 it counts as normal first aid, plus one of these, your choice:
\holdoptions%
    {{Stabilize the injury but the patient takes -1 forward.},
    {Heal 1-harm and stabilize for now, but it will return as 2-harm and become unstable again later.},
    {Heal 1-harm and stabilize but the patient takes -1 ongoing until it’s fixed properly.}}
\end{move}%
\br[1mm]
%
\begin{move}{Often Overlooked}
When you \textbf{act all crazy} to avoid something, roll +Weird.  On a 10+ you’re regarded as unthreatening and unimportant.  On a 7-9, pick one: unthreatening or unimportant.  On a miss you draw lots (but not all) of the attention.
\end{move}%
\br[1mm]
%
\begin{move}{Its Not as Bad As It Looks}
Once per mystery \textbf{you may attempt to resist all paranormal influences} affecting you.  Roll +Cool.  On a 10+ heal 2 harm and you are resilient against anything paranormal, for a time.  On a 7-9 you may either break free of a paranormal influence, or heal 2 harm.  On a miss it’s worse than you thought, the paranormal influence takes over, the Keeper will say how.
\end{move}%
\br[1mm]
%
\begin{move}{Contrary}
When you \textbf{seek out and receive someone’s honest advice} on the best course of action for you and then do something else instead, mark experience. If you do exactly the opposite of their advice, you also take +1 ongoing on any moves you make pursuing that course.
\end{move}
%
\begin{move}{Fellow Believer}
You’re relatable in ways that other Agents aren’t.  People see you as a fellow conspiracy theorist, and are willing to open up to you about weird things they wouldn’t mention around other government agents.
\end{move}
%
}\pagebreak% -- back page
%
\playbookpage{%
%
% -- pg.2/2 col.1/3
%
\begin{pbsect}{GEAR}[]
You have protective wear, suited to your look, worth 1-armour, a 9mm (2-harm close loud) and a big knife (1-harm hand).
\end{pbsect}%
\br[2mm]
%
\optionsparen{Non-Bureau Approved Protection}{choose two}%
    {{\textbf{Tinfoil Hat:} while worn, you are shielded from direct paranormal influence on your mind.  You also look like a nut, and take -1 Charm ongoing.},
    {\textbf{Lead-lined Vest:} while worn, psychic harm against you cannot ignore armor.  You also insulate yourself from paranormal abilities, take -1 Weird ongoing.},
    {\textbf{Rubber Biohazard Suit:} while worn, and in good condition, you are protected from any biological effects such as molds and toxins.  You also have a difficult time moving, take -1 Tough ongoing.},
    {\textbf{Modified Welding Mask:} while worn, you are unaffected by paranormal effects triggered when looking at, or away from, something.  You have a difficult time seeing anything, really, and take -1 Sharp ongoing.},
    {\textbf{Resonance Cancelling Headphones:} while worn, auditory paranormal effects cannot affect you.  You barely hear normal sounds either, and take -1 Cool ongoing.}}%
\brln
%
\begin{pbsect}{EXPERT DRIVER}
You have +1 ongoing while driving.  Plus you can hotwire any vehicle (the older and more normal it is, the fewer tools you need to do it). You also have two vehicles of your own.
\optionsparen{Vehicle}{choose two}%
    {Classic motorcycle,
    Dirt Bike,
    Classic car,
    Nondescript car,
    Pickup truck,
    Cargo van,
    Minivan}
\end{pbsect}
%
}{% -- pg.2/2 col.2/3
%
\optionsparen{Ratings}{pick one line}%
    {{Charm=0, Cool+1, Sharp-1, Tough+2, Weird+1},
    {Charm=0, Cool=0, Sharp+1, Tough+2, Weird=0},
    {Charm+1, Cool=0, Sharp+1, Tough+2, Weird-1},
    {Charm-1, Cool-1, Sharp=0, Tough+2, Weird+2},
    {Charm+1, Cool-1, Sharp=0, Tough+2, Weird+1}}
%
\introductions{Outsider}
%
\history%
    {{They helped you at a critical point in your quest to find the truth. Tell them what you needed help with.},
    {They stood between you and what you needed to find out. Ask them why.},
    {They also lost a friend or relative to the paranormal. Ask them who it was.},
    {Relations, close or distant. Tell them exactly what.},
    {You respect their hard-earned knowledge, and often come to them for advice.},
    {They introduced you to the Bureau when you were learning the truth about the paranormal.},
    {They saw you absolutely lose it and go unhinged. Tell them what the situation was, and ask them how much collateral damage you caused.}}%
%
}{% -- pg.2/2 col.3/3
%
\levelingup
%
\improvementsonecol{%
    {Get +1 Tough, max +3.},
    {Get +1 Weird, max +3.},
    {Get +1 Cool, max +3.},
    {Get +1 Sharp, max +3.},
    {Take another Outsider move.},
    {Take another Outsider move.},
    {Take a move from another playbook.},
    {Take a move from another playbook.},
    {Gain an Astral Guide, like the Psychic has, and one of their Power moves.},
    {Gain another of the Psychic’s Power moves.}
}{%
    {Get +1 to any rating, max +3.},
    {Get back one used Luck point.},
    {Create a second Bureau Agent to play as well as this one.},
    {Mark two of the basic moves as advanced.},
    {Mark another two of the basic moves as advanced.},
    {Jump departments: change to a new playbook.},
    {Get promoted to the Head of Containment.  This character becomes an NPC.  Start a new character.},
    {Retire to safety.},
    {Resolve your first encounter. The Keeper makes the next mystery about this event, and should try to answer all remaining questions about it during the mystery (although there are sure to be new threads to investigate after...)}
}%
%
}%
% -- end playbook
\pagebreak

% The Psychic ---------------------------------------
%%%%%%%%%%%%%%%%%%%%%%%%%%%%%%%%%%%%%%%%%%%%%%%%%%%%%%%%%%%%%%%%%%%%%%%%%%%%%%%%%%%%%%%%
%
% Altered World Event of the Week is a Monster of the Week derivative.
%
% Much of the textual content in the playbooks is copied from the original playbooks,
% following approval by Michael Sands for the free distribution of the AWEotW document.
% See: https://genericgames.co.nz/third_party_policy/ for his third party policy.
%
% Re-use of this script is for personal reproduction of the formatting and generation
% of other custom playbooks.  Any reproduction of the text should seek approval per the
% aforementioned generic games third party policy.
%
% Monster of the Week is copyrighted by Evil Hat Productions, LLC and Generic Games.
%
%%%%%%%%%%%%%%%%%%%%%%%%%%%%%%%%%%%%%%%%%%%%%%%%%%%%%%%%%%%%%%%%%%%%%%%%%%%%%%%%%%%%%%%%

% the psychic ---------------------------
%
% -- front page
%
\playbookpage{%
%
% -- pg.1/2 col.1/3
%
\pbcommon{Psychic}%
{I didn't choose to come here.  Something brought me.  Leading me every step of the way.  Tugging on my strings like I'm some kind of puppet.}%
{As you mark off your Luck boxes, your Astral Guide will get more demanding.}%
%
}{% -- pg.1/2 col.2/3
%
\moveexp{four}{Director}
\phantom{.}\\
%
\textit{Pick two of the four Power moves below.  These moves cannot be taken when leveling up unless specifically allowed (or your Keeper agrees).}\\
%
\begin{move}{Power: Psychic Whammy}
You can use your powers to \textbf{kick some ass}: roll +Weird instead of +Tough. The attack has 2-harm hand obvious ignore-armour. On a miss, you’ll get a psychic backlash
\end{move}%
\br[1mm]
%
\begin{move}{Power: Launch}
You can use your powers to levitate and throw objects with incredible force.  When you use launch to \textbf{kick some ass}: roll +Weird instead of +Tough.  The attack has 2-harm close obvious loud.  On a miss, you cause unintended collateral damage or put someone in danger.
\end{move}%
\br[1mm]
%
\begin{move}{Power: Shield}
When you use your powers to \textbf{protect someone} by forming a barrier with nearby objects, you gain 2-armour against any harm that is transferred to you. This doesn’t stack with your other armour, if any.
\end{move}%
\br[1mm]
%
\begin{move}{Power: Levitate}
You can quickly float up into the air, twenty feet or so, and hover there for a couple moments.  Eventually gravity will bring you back down, albeit very slowly.  Anytime you’re falling, you can control the rate at which you fall.  It’s not exactly the same as flying, but it’s as good as anyone else has ever done.
\end{move}%
\br[4mm]
%
\textit{Then pick two of these:}\\
%
\begin{move}{Telepathy}
You can read people’s thoughts and put words in their mind. This can allow you to \textbf{investigate a mystery} or read a bad situation without needing to actually talk. You can also manipulate someone without speaking. You still roll moves as normal, except people will not expect the weirdness of your mental communication.
\end{move}
%
}{% -- pg.1/2 col.3/3
%
\begin{move}{The Sight}
You can see the invisible, especially paranormal entities and influences. You may communicate with (maybe even make deals with) the paranormal forces you see, and they give you more opportunities to spot clues when you \textbf{investigate a mystery}.
\end{move}%
\br[1mm]
%
\begin{move}{Hunches}
\textbf{When something bad is happening} (or just about to happen) somewhere that you aren’t, roll +Sharp. On a 10+ you knew where you needed to go, just in time to get there. On a 7-9, you get there late—in time to intervene, but not prevent it altogether. On a miss, you get there just in time to be in trouble yourself.
\end{move}%
\br[1mm]
%
\begin{move}{Jinx}
You can \textbf{encourage coincidences to occur}, the way you want. When you jinx a target, roll +Weird. On a 10+ hold 2 and on a 7-9 hold 1. On a miss, the Keeper holds 2 over you to be used in the same way. Spend your hold to:
\holdoptions%
    {{Interfere with a bureau agent, giving them -1 forward.},
    {Help an agent, giving them +1 forward, by interfering with their enemy.},
    {Interfere with what a paranormal entity or object, or a bystander is trying to do.},
    {Inflict 1-harm on the target due to an accident.},
    {The target finds something you left for them.},
    {The target loses something that you will soon find.}}
\end{move}%
\br[1mm]
%
\begin{move}{Subdue}
When you \textbf{concentrate on taking control of a paranormally affected item or person} you may soothe or cleanse your target. Roll +Tough.  On a 10+ you get the effect you want.  On a 7-9 it takes a little while before it takes effect.  The paranormal presence has time to take one or two actions.  On a miss, something is preventing you from getting through to it.  That’s bad. 
\end{move}
%
}\pagebreak% -- back page
%
\playbookpage{%
%
% -- pg.2/2 col.1/3
%
\begin{move}{Third Eye}
When you \textbf{read a bad situation}, you can open up your third eye for a moment to take in extra information. Take +1 hold on any result of 7 or more, plus you can see invisible things. On a miss, you may still get 1 hold, but you’re exposed to supernatural danger. Unfiltered hidden reality is rough on the mind!
\end{move}
%
\begin{pbsect}{ASTRAL GUIDE}
Your powers have a source that is with you always, guiding you through the world.  It’s like a resonance, or a being in your dreams, or a flash of images and words inside your head.  It might be helping you because it likes you, or it might be driving you to pursue its own wants and goals.  Something like that.  Whatever, you barely understand it either.
\end{pbsect}%
\br[2mm]
%
\optionsparen{Signals from your Guide}{pick one}%
    {{A kaleidoscopic image that flashes through your vision.},
    {A creature who visits you in your head.},
    {A sudden urge to chant certain phrases.},
    {A disembodied voice talking to you in disjointed phrases.},
    {A vision from somewhere in the astral plane.}}%
\br[2mm]
The keeper can inject signals into the world as a sign that your source would like you to follow that path.  If you go where the signal wants you to, mark experience.  If you don’t, then your powers are unavailable until the end of the mystery (or until you cave).  As you mark off luck boxes, your guide will become pushier about getting you where it wants you to be.
%
\begin{pbsect}{TWO WAY COMMUNICATION}
\textbf{At the start of each mystery}, hold 2.  You can spend this hold anytime to send a message telepathically to your guide.  When you do, roll +Weird.  On a 10+, your guide reveals a key fact, clue, or location that will help you.  On a 7-9, the answer is clouded or difficult to interpret.  On a miss your guide’s reply is terrible, garbled, or somehow wrong.
\end{pbsect}
%
}{% -- pg.2/2 col.2/3
%
\optionsparen{Ratings}{pick one line}%
    {{Charm+1, Cool=0, Sharp+1, Tough-1, Weird+2},
    {Charm-1, Cool+1, Sharp=0, Tough+1, Weird+2},
    {Charm+2, Cool=0, Sharp-1, Tough-1, Weird+2},
    {Charm=0, Cool-1, Sharp+1, Tough+1, Weird+2},
    {Charm-1, Cool-1, Sharp+2, Tough=0, Weird+2}}
%
\introductions{Psychic}
%
\begin{history}
\begin{itemize}
\item They taught you to control your powers, to the extent that you can control them at all.
\item You are blood-kin. Decide together exactly what.
\item You are married, or romantically involved. Decide between you the exact relationship.
\item You’re old friends, and trust each other completely.
\item You used your powers on them one time. Decide if it was for selfish reasons or not, and tell them if they found out about it.
\item You’ve known each other some time, but since your powers manifested, you keep them at a distance emotionally.
\item You hope they can help you control your powers.
\item They saw you use your powers for selfish or vindictive reasons. Ask them who the victim was, and then tell them what you did.
\end{itemize}
\end{history}
%
}{% -- pg.2/2 col.3/3
%
\levelingup
%
\improvementsonecol{%
    {Get +1 Charm, max +3.},
    {Get +1 Weird, max +3.},
    {Get +1 Tough, max +3.},
    {Get +1 Sharp, max +3.},
    {Take another Psychic Power move.},
    {Take another Psychic or Psychic Power move.},
    {Take another Psychic move.},
    {Take a move from another playbook.},
    {Take a move from another playbook.},
    {You can use the Director’s Hotline.}
}{%
    {Get +1 to any rating, max +3.},
    {Get back one used Luck point.},
    {Create a second Bureau Agent to play as well as this one.},
    {Mark two of the basic moves as advanced.},
    {Mark another two of the basic moves as advanced.},
    {Get promoted somewhere in the Bureau.  This character becomes an NPC.  Start a new character.},
    {Retire to safety.},
    {Resolve the mystery of your Astral Guide. The Keeper makes the next mystery about this event, and should try to answer all remaining questions about it during the mystery (although there are sure to be new threads to investigate after).  Whether you keep your Guide and your powers depends on how the mystery gets resolved...}
}%
%
}%
% -- end playbook
\pagebreak

% Links ---------------------------------------
\fontsize{12}{14}\selectfont
\chapter*{Useful Links}
\begin{itemize}

\item \textbf{Remedy’s Control:} \url{https://www.remedygames.com/games/control/}
\item \textbf{SCP Foundation:} \url{http://www.scpwiki.com/}
\item \textbf{About Monster of the Week:} \url{https://www.genericgames.co.nz/motw/}
\item \textbf{MotW resources:} \url{https://www.evilhat.com/home/monster-of-the-week-resources/}
\item \textbf{MotW More Weirdness:} \url{https://genericgames.co.nz/files/MotW_more_weirdness.pdf/}
\item \textbf{{\LaTeX} source code for this PDF:} \url{https://github.com/rfkeepers/aweotw-tex}

\end{itemize}
\pagebreak
\end{document}