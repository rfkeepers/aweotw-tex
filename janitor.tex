%%%%%%%%%%%%%%%%%%%%%%%%%%%%%%%%%%%%%%%%%%%%%%%%%%%%%%%%%%%%%%%%%%%%%%%%%%%%%%%%%%%%%%%%
%
% Altered World Event of the Week is a Monster of the Week derivative.
%
% Much of the textual content in the playbooks is copied from the original playbooks,
% following approval by Michael Sands for the free distribution of the AWEotW document.
% See: https://genericgames.co.nz/third_party_policy/ for his third party policy.
%
% Re-use of this script is for personal reproduction of the formatting and generation
% of other custom playbooks.  Any reproduction of the text should seek approval per the
% aforementioned generic games third party policy.
%
% Monster of the Week is copyrighted by Evil Hat Productions, LLC and Generic Games.
%
%%%%%%%%%%%%%%%%%%%%%%%%%%%%%%%%%%%%%%%%%%%%%%%%%%%%%%%%%%%%%%%%%%%%%%%%%%%%%%%%%%%%%%%%

% the janitor ---------------------------
%
% -- front page
%
\playbookpage{%
%
% -- pg.1/2 col.1/3
%
\pbcommon{Janitor}%
{Someone called looking for an assistant.  I answered the call.  They let me into the building, showed me around the place.  I've been here ever since.}%
{When you spend a point of luck, the Keeper will cause the oldest house to shift, or release a tiresome pest somewhere within.}%
%
}{% -- pg.1/2 col.2/3
%
\moveexp{four}{Janitor}%
\br[2mm]
%
\textit{Pick two of the following three.  You cannot pick the third when you level up.  Only the Janitor can pick these moves.}
%
\br[1mm]
%
\begin{move}{Supply Closets}
When you’re inside the Bureau, or another building you’ve previously worked within, and need something that you could conceivably find in your janitorial supplies, there is a nearby supply closet that has it.
\br[2mm]
If you \textbf{need something that isn’t normally found in janitorial supplies}, open the nearest supply closet and roll +Weird.  On a 10+ it’s there, just like you needed.  On a miss, there’s something in the closet you aren’t going to like.  On a 7-9, both.
\end{move}%
\br[2mm]
%
\begin{move}{The Janitor Always Has The Keys}
When you need access to a part of the Bureau, or another building you’ve previously worked within, you either already have a set of keys that unlock the doors, or you know an alternate way through the building to get to the same spot, the Keeper will tell you which one.
\end{move}%
\br[2mm]
%
\begin{move}{Rule of Three}
Add the following option to \textbf{Housekeeper}.
\holdoptions{{A lightswitch cord appears somewhere nearby, say where.  Ask or tell the Keeper what will be waiting for you when you use it, and what will have changed when you return.}}
\end{move}
%
}{% -- pg.1/2 col.3/3
%
\textit{Then pick two of these:}\\
%
\begin{move}{Oops!}
If you want to \textbf{stumble across something important}, tell the Keeper. You will find something important and useful, although not necessarily related to your immediate problems.
\end{move}%
\br[1mm]
%
\begin{move}{Let's Get Out of Here!}
If you can \textbf{protect someone} by telling them what to do, or by leading them out, roll +Charm instead of +Tough.
\end{move}%
\br[1mm]
%
\begin{move}{Through the Air Ducts}
When \textbf{you need to escape}, name the route you’ll try and roll +Sharp. On a 10+ you’re out of danger, no problem. On a 7-9 you can go or stay, but if you go it’s going to cost you (you leave some-thing behind or something comes with you). On a miss, you are caught halfway out.
\end{move}%
\br[1mm]
%
\begin{move}{The Power of Heart}
When fighting a monster, if you \textbf{help someone}, don’t roll +Cool. You automatically help as though you’d rolled a 10.
\end{move}%
\br[1mm]
%
\begin{move}{Don't Worry, I'll Check It Out}
Whenever you go off by yourself to check out somewhere (or something) scary or dangerous, mark experience.
\end{move}%
\br[1mm]
%
\begin{move}{So Much Work To Do}
While inside the oldest house, you heal faster than normal people. Any time your harm gets healed, heal an extra point. You are immune to all the harm move effects under ‘0-harm’ and ‘1-harm’ (when the Keeper would apply these, you ignore it).
\end{move}%
\br[1mm]
%
\begin{move}{Just Another Day}
When you have to \textbf{act under pressure} due to an altered item, phenomenon, or paranormal effect, you may roll +Weird instead of +Cool.
\end{move}%
\br[1mm]
%
}\pagebreak% -- back page
%
\playbookpage{%
%
% -- pg.2/2 col.1/3
%
\begin{pbsect}{TALISMAN}[]
You have an Object of Power that helps or protects you. Define it, and its power, with the Keeper’s agreement. The object is one of: a janitorial supply item, a discrete and old piece of electronics, or something that plays music.  The Bureau is not aware that this is an object of power, or that you have objects of power in your possession.
\end{pbsect}
\br[2mm]
%
\begin{pbsect}{HOUSEKEEPER}
You may not always know what the people in the Bureau are doing, or why they’re doing it, but no one knows the oldest house itself like you do.  Whenever you ask the Keeper where a place or thing is located inside the house, even if that location is supposed to be secret, even if the house has shifted, they will tell you where it is (or is supposed to be).
%
\begin{center}
Knowledge: \checkbox[7]
\end{center}
%
At any time during play you may mark a box of Knowledge to do one of the following:
\holdoptions
    {{Describe a sector or threshold within the oldest house.  That place now exists.  With the Keeper’s agreement, describe details such as whether it is overt or secret, in use or vacated, and what can be found there.},
    {Ask the Keeper about any paranormal thing that exists inside the oldest house.  They will tell you three things about that item.},
    {Ask or tell the Keeper about a ritual or procedure employees use to control or work within the oldest house.  You now know as much as any other bureau employee about it, and are able to reproduce the effects.}}
\end{pbsect}%
%
}{% -- pg.2/2 col.2/3
%
\optionsparen{Ratings}{pick one line}%
    {{Charm+2, Cool+1, Sharp=0, Tough+1, Weird-1},
    {Charm+2, Cool-1, Sharp+1, Tough+1, Weird=0},
    {Charm+2, Cool=0, Sharp-1, Tough+1, Weird+1},
    {Charm+2, Cool=0, Sharp+1, Tough+1, Weird-1},
    {Charm+2, Cool+1, Sharp+1, Tough=0, Weird-1}}
%
\introductions{Janitor}
\brln
%
\begin{history}
\begin{itemize}
\item You are close relations. Tell them exactly how you’re related.
\item You helped get them their job in the Bureau.  Tell them what strings you pulled.
\item Workplace buddies.  Your responsibilities caused you to bump into each other frequently and that developed into a lasting friendship.
\item They’ve been breaking the Bureau’s rules and they think they’re getting away with it.  But you’ve seen them do it.  Ask them what you’ve seen them do.
\item You’re a bit suspicious, or extra welcoming, of them (maybe due to their unnatural powers or something like that).
\item You accidentally ruined some of their work while cleaning up their space.  Ask them what happened.
\item Due to an unlikely chain of events you saved their life from a safety breach within the Bureau. Tell them how.
\end{itemize}
\end{history}
%
}{% -- pg.2/2 col.3/3
%
\levelingup
%
\improvementsonecol{%
    {Get +1 Charm, max +3.},
    {Get +1 Cool, max +3.},
    {Get +1 Weird, max +3.},
    {Get +1 Tough, max +3.},
    {Take another Janitor move.},
    {Take another Janitor move.},
    {Take a move from another playbook.},
    {Take a move from another playbook.},
    {Take a move from another playbook.},
    {Get back one used Knowledge or Luck point.},
    {Get back one used Knowledge or Luck point.}
}{%
    {Get +1 to any rating, max +3.},
    {Get back one used Knowledge and Luck point.},
    {Get back one used Knowledge and Luck point.},
    {Mark two of the basic moves as advanced.},
    {Mark another two of the basic moves as advanced.}
}%
%
}%
% -- end playbook